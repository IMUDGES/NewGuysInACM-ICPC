\chapter*{前言}

\section*{ACM与ICPC}

ACM/ICPC是有美国计算机协会(ACM)主办的“ACM国际大学生程序设计竞赛”。这是一场展示大学生创新能力、团队精神和在压力下编写程序、分析和解决问题能力的年度竞赛,已经有30多年的历史。是计算机领域大学生竞赛中颇具含金量的一项赛事!

ACM/ICPC以团队的形式代表各学校参加比赛,每队由3名队员组成。比赛期间,每队使用1台电脑,在规定的时间内使用规定的编程语言完成规定的问题。程序编写完成后提交给裁判处运行,运行结果将会随时返回给队员。期间如果提交答案被判定为错误,会被罚在总时间上增加20分钟罚时。若该题最终回答正确,则对题数加1,同时罚时计入总时间。如果直至比赛结束也未能最对,则该题不计时。

比赛最终的获胜者为解答题目数量最多的队伍。当数目一样多的时候,则按照完成问题所用时间排名。

与IOI等其他的计算机程序设计竞赛相比,ACM/ICPC的特点就是题量大,队员多,电脑少。所以除了扎实的专业水平,良好的团队协作和优秀的心理素质也是非常重要的。

参赛队需要通过预选赛层层选拔,过五关斩六将,才有机会代表参加全球总决赛。并且获得丰厚的奖励。

ACM/ICPC应该算是大学生所能参加比赛中含金量最高的比赛了,因为它真正考研了参赛选手的各方面的综合素质,而不仅仅是背书的能力。只有通过大量的训练、锻炼出优秀的计算思维才能取得优秀的成绩。

\section*{关于本书}

本书是为内蒙古大学计算机学院2013级ACM/ICPC新手训练营所编写的系列教程。编写者是一群来自于内蒙古大学精英学生开发者联盟的靠谱青年。

本教程力求用“不靠谱”的方式去介绍很多“靠谱”的事情。所以我们会在不影响大家阅读兴趣的情况下,尽量保证内容的严谨性!

关于书中内容:
第零章的绪论,为大家介绍了一些学习计算机程序设计所必备的基础知识。
第一章的主要内容是C++程序设计语言的快速入门。
第二章将会为大家讲解一些非常简单,但不失趣味性的入门算法。
第三章主要是各种常见、常用的排序算法。
第四章将会为大家仔细介绍几种数据结构——用来对现实问题进行抽象描述的工具。
第五章将会讲解一些参加ACM/ICPC这样类型比赛的一些基本算法。
第六章将会讲解一些诸如动态规划、图论等的高级算法。
第七章将主要会围绕着计算机算法领域的很多数学算法进行讲解。

当然,写书并不是一件简单的事情,尤其是对一帮平均年龄还大(bú)约(dào)20岁的大学生来说。所以书中难(kěn)免(dìng)会出现各种错误。希望大家在看到书中错误的时候能够及时指出!

本书的编写者有(按照章节编写顺序):石博天、张怀文、李政霖、薛家斌、杜志浩。本书勘误请联系邮箱:contact@imudges.com。希望本书能够打开大家走向高手之路的大门! 
